\documentclass[]{article}
\usepackage{lmodern}
\usepackage{amssymb,amsmath}
\usepackage{ifxetex,ifluatex}
\usepackage{fixltx2e} % provides \textsubscript
\ifnum 0\ifxetex 1\fi\ifluatex 1\fi=0 % if pdftex
  \usepackage[T1]{fontenc}
  \usepackage[utf8]{inputenc}
\else % if luatex or xelatex
  \ifxetex
    \usepackage{mathspec}
  \else
    \usepackage{fontspec}
  \fi
  \defaultfontfeatures{Ligatures=TeX,Scale=MatchLowercase}
\fi
% use upquote if available, for straight quotes in verbatim environments
\IfFileExists{upquote.sty}{\usepackage{upquote}}{}
% use microtype if available
\IfFileExists{microtype.sty}{%
\usepackage{microtype}
\UseMicrotypeSet[protrusion]{basicmath} % disable protrusion for tt fonts
}{}
\usepackage[margin=1in]{geometry}
\usepackage{hyperref}
\hypersetup{unicode=true,
            pdftitle={Homework 5},
            pdfauthor={Noah Kawasaki},
            pdfborder={0 0 0},
            breaklinks=true}
\urlstyle{same}  % don't use monospace font for urls
\usepackage{color}
\usepackage{fancyvrb}
\newcommand{\VerbBar}{|}
\newcommand{\VERB}{\Verb[commandchars=\\\{\}]}
\DefineVerbatimEnvironment{Highlighting}{Verbatim}{commandchars=\\\{\}}
% Add ',fontsize=\small' for more characters per line
\usepackage{framed}
\definecolor{shadecolor}{RGB}{248,248,248}
\newenvironment{Shaded}{\begin{snugshade}}{\end{snugshade}}
\newcommand{\KeywordTok}[1]{\textcolor[rgb]{0.13,0.29,0.53}{\textbf{#1}}}
\newcommand{\DataTypeTok}[1]{\textcolor[rgb]{0.13,0.29,0.53}{#1}}
\newcommand{\DecValTok}[1]{\textcolor[rgb]{0.00,0.00,0.81}{#1}}
\newcommand{\BaseNTok}[1]{\textcolor[rgb]{0.00,0.00,0.81}{#1}}
\newcommand{\FloatTok}[1]{\textcolor[rgb]{0.00,0.00,0.81}{#1}}
\newcommand{\ConstantTok}[1]{\textcolor[rgb]{0.00,0.00,0.00}{#1}}
\newcommand{\CharTok}[1]{\textcolor[rgb]{0.31,0.60,0.02}{#1}}
\newcommand{\SpecialCharTok}[1]{\textcolor[rgb]{0.00,0.00,0.00}{#1}}
\newcommand{\StringTok}[1]{\textcolor[rgb]{0.31,0.60,0.02}{#1}}
\newcommand{\VerbatimStringTok}[1]{\textcolor[rgb]{0.31,0.60,0.02}{#1}}
\newcommand{\SpecialStringTok}[1]{\textcolor[rgb]{0.31,0.60,0.02}{#1}}
\newcommand{\ImportTok}[1]{#1}
\newcommand{\CommentTok}[1]{\textcolor[rgb]{0.56,0.35,0.01}{\textit{#1}}}
\newcommand{\DocumentationTok}[1]{\textcolor[rgb]{0.56,0.35,0.01}{\textbf{\textit{#1}}}}
\newcommand{\AnnotationTok}[1]{\textcolor[rgb]{0.56,0.35,0.01}{\textbf{\textit{#1}}}}
\newcommand{\CommentVarTok}[1]{\textcolor[rgb]{0.56,0.35,0.01}{\textbf{\textit{#1}}}}
\newcommand{\OtherTok}[1]{\textcolor[rgb]{0.56,0.35,0.01}{#1}}
\newcommand{\FunctionTok}[1]{\textcolor[rgb]{0.00,0.00,0.00}{#1}}
\newcommand{\VariableTok}[1]{\textcolor[rgb]{0.00,0.00,0.00}{#1}}
\newcommand{\ControlFlowTok}[1]{\textcolor[rgb]{0.13,0.29,0.53}{\textbf{#1}}}
\newcommand{\OperatorTok}[1]{\textcolor[rgb]{0.81,0.36,0.00}{\textbf{#1}}}
\newcommand{\BuiltInTok}[1]{#1}
\newcommand{\ExtensionTok}[1]{#1}
\newcommand{\PreprocessorTok}[1]{\textcolor[rgb]{0.56,0.35,0.01}{\textit{#1}}}
\newcommand{\AttributeTok}[1]{\textcolor[rgb]{0.77,0.63,0.00}{#1}}
\newcommand{\RegionMarkerTok}[1]{#1}
\newcommand{\InformationTok}[1]{\textcolor[rgb]{0.56,0.35,0.01}{\textbf{\textit{#1}}}}
\newcommand{\WarningTok}[1]{\textcolor[rgb]{0.56,0.35,0.01}{\textbf{\textit{#1}}}}
\newcommand{\AlertTok}[1]{\textcolor[rgb]{0.94,0.16,0.16}{#1}}
\newcommand{\ErrorTok}[1]{\textcolor[rgb]{0.64,0.00,0.00}{\textbf{#1}}}
\newcommand{\NormalTok}[1]{#1}
\usepackage{graphicx,grffile}
\makeatletter
\def\maxwidth{\ifdim\Gin@nat@width>\linewidth\linewidth\else\Gin@nat@width\fi}
\def\maxheight{\ifdim\Gin@nat@height>\textheight\textheight\else\Gin@nat@height\fi}
\makeatother
% Scale images if necessary, so that they will not overflow the page
% margins by default, and it is still possible to overwrite the defaults
% using explicit options in \includegraphics[width, height, ...]{}
\setkeys{Gin}{width=\maxwidth,height=\maxheight,keepaspectratio}
\IfFileExists{parskip.sty}{%
\usepackage{parskip}
}{% else
\setlength{\parindent}{0pt}
\setlength{\parskip}{6pt plus 2pt minus 1pt}
}
\setlength{\emergencystretch}{3em}  % prevent overfull lines
\providecommand{\tightlist}{%
  \setlength{\itemsep}{0pt}\setlength{\parskip}{0pt}}
\setcounter{secnumdepth}{0}
% Redefines (sub)paragraphs to behave more like sections
\ifx\paragraph\undefined\else
\let\oldparagraph\paragraph
\renewcommand{\paragraph}[1]{\oldparagraph{#1}\mbox{}}
\fi
\ifx\subparagraph\undefined\else
\let\oldsubparagraph\subparagraph
\renewcommand{\subparagraph}[1]{\oldsubparagraph{#1}\mbox{}}
\fi

%%% Use protect on footnotes to avoid problems with footnotes in titles
\let\rmarkdownfootnote\footnote%
\def\footnote{\protect\rmarkdownfootnote}

%%% Change title format to be more compact
\usepackage{titling}

% Create subtitle command for use in maketitle
\newcommand{\subtitle}[1]{
  \posttitle{
    \begin{center}\large#1\end{center}
    }
}

\setlength{\droptitle}{-2em}
  \title{Homework 5}
  \pretitle{\vspace{\droptitle}\centering\huge}
  \posttitle{\par}
  \author{Noah Kawasaki}
  \preauthor{\centering\large\emph}
  \postauthor{\par}
  \predate{\centering\large\emph}
  \postdate{\par}
  \date{6/6/2018}


\begin{document}
\maketitle

\subsection{Textbook A}\label{textbook-a}

\subsection{Problem 14.2}\label{problem-14.2}

\begin{Shaded}
\begin{Highlighting}[]
\NormalTok{m1 <-}\StringTok{ }\KeywordTok{garch.sim}\NormalTok{(}\DataTypeTok{alpha=}\KeywordTok{c}\NormalTok{(}\DecValTok{3}\NormalTok{, }\FloatTok{0.35}\NormalTok{), }\DataTypeTok{beta=}\KeywordTok{c}\NormalTok{(}\FloatTok{0.6}\NormalTok{), }\DataTypeTok{n=}\DecValTok{500}\NormalTok{)  }\CommentTok{# 0.75}
\NormalTok{mean1 <-}\StringTok{ }\KeywordTok{mean}\NormalTok{(m1)}
\NormalTok{sd1 <-}\StringTok{ }\KeywordTok{sd}\NormalTok{(m1)}
\NormalTok{sm1 <-}\StringTok{ }\NormalTok{(m1}\OperatorTok{-}\NormalTok{mean1)}\OperatorTok{/}\NormalTok{sd1}
\NormalTok{sds1 <-}\StringTok{ }\KeywordTok{sqrt}\NormalTok{((m1}\OperatorTok{-}\NormalTok{mean1)}\OperatorTok{^}\DecValTok{2}\NormalTok{)}

\NormalTok{m2 <-}\StringTok{ }\KeywordTok{garch.sim}\NormalTok{(}\DataTypeTok{alpha=}\KeywordTok{c}\NormalTok{(}\DecValTok{3}\NormalTok{, }\FloatTok{0.1}\NormalTok{), }\DataTypeTok{beta=}\KeywordTok{c}\NormalTok{(}\FloatTok{0.6}\NormalTok{), }\DataTypeTok{n=}\DecValTok{500}\NormalTok{)  }\CommentTok{# 0.16}
\NormalTok{mean2 <-}\StringTok{ }\KeywordTok{mean}\NormalTok{(m2)}
\NormalTok{sd2 <-}\StringTok{ }\KeywordTok{sd}\NormalTok{(m2)}
\NormalTok{sm2 <-}\StringTok{ }\NormalTok{(m2}\OperatorTok{-}\NormalTok{mean2)}\OperatorTok{/}\NormalTok{sd2}
\NormalTok{sds2 <-}\StringTok{ }\KeywordTok{sqrt}\NormalTok{((m2}\OperatorTok{-}\NormalTok{mean2)}\OperatorTok{^}\DecValTok{2}\NormalTok{)}

\CommentTok{# Time Series}
\KeywordTok{ggplot}\NormalTok{(}\KeywordTok{data.frame}\NormalTok{(m1), }\KeywordTok{aes}\NormalTok{(}\DataTypeTok{x=}\KeywordTok{seq}\NormalTok{(}\DataTypeTok{from=}\DecValTok{1}\NormalTok{, }\DataTypeTok{to=}\KeywordTok{length}\NormalTok{(m1)), }\DataTypeTok{y=}\NormalTok{m1)) }\OperatorTok{+}
\StringTok{  }\KeywordTok{geom_line}\NormalTok{(}\DataTypeTok{color=}\StringTok{"red"}\NormalTok{) }\OperatorTok{+}
\StringTok{  }\KeywordTok{ylim}\NormalTok{(}\OperatorTok{-}\DecValTok{18}\NormalTok{, }\DecValTok{18}\NormalTok{) }\OperatorTok{+}
\StringTok{  }\KeywordTok{ggtitle}\NormalTok{(}\StringTok{"GARCH(1,1)"}\NormalTok{, }\StringTok{"High Persistence"}\NormalTok{) }\OperatorTok{+}
\StringTok{  }\KeywordTok{xlab}\NormalTok{(}\StringTok{"Index"}\NormalTok{) }\OperatorTok{+}
\StringTok{  }\KeywordTok{ylab}\NormalTok{(}\StringTok{"Value"}\NormalTok{)}
\end{Highlighting}
\end{Shaded}

\includegraphics{homework_5_markdown_files/figure-latex/unnamed-chunk-1-1.pdf}

\begin{Shaded}
\begin{Highlighting}[]
\KeywordTok{ggplot}\NormalTok{(}\KeywordTok{data.frame}\NormalTok{(m2), }\KeywordTok{aes}\NormalTok{(}\DataTypeTok{x=}\KeywordTok{seq}\NormalTok{(}\DataTypeTok{from=}\DecValTok{1}\NormalTok{, }\DataTypeTok{to=}\KeywordTok{length}\NormalTok{(m2)), }\DataTypeTok{y=}\NormalTok{m2)) }\OperatorTok{+}
\StringTok{  }\KeywordTok{geom_line}\NormalTok{(}\DataTypeTok{color=}\StringTok{"blue"}\NormalTok{) }\OperatorTok{+}
\StringTok{  }\KeywordTok{ylim}\NormalTok{(}\OperatorTok{-}\DecValTok{18}\NormalTok{, }\DecValTok{18}\NormalTok{) }\OperatorTok{+}
\StringTok{  }\KeywordTok{ggtitle}\NormalTok{(}\StringTok{"GARCH(1,1)"}\NormalTok{, }\StringTok{"Low Persistence"}\NormalTok{) }\OperatorTok{+}
\StringTok{  }\KeywordTok{xlab}\NormalTok{(}\StringTok{"Index"}\NormalTok{) }\OperatorTok{+}
\StringTok{  }\KeywordTok{ylab}\NormalTok{(}\StringTok{"Value"}\NormalTok{)}
\end{Highlighting}
\end{Shaded}

\includegraphics{homework_5_markdown_files/figure-latex/unnamed-chunk-1-2.pdf}

\begin{Shaded}
\begin{Highlighting}[]
\CommentTok{# Conditional Standard Deviations}
\KeywordTok{ggplot}\NormalTok{(}\KeywordTok{data.frame}\NormalTok{(m1), }\KeywordTok{aes}\NormalTok{(}\DataTypeTok{x=}\KeywordTok{seq}\NormalTok{(}\DataTypeTok{from=}\DecValTok{1}\NormalTok{, }\DataTypeTok{to=}\KeywordTok{length}\NormalTok{(m1)), }\DataTypeTok{y=}\NormalTok{sds1)) }\OperatorTok{+}
\StringTok{  }\KeywordTok{geom_line}\NormalTok{(}\DataTypeTok{color=}\StringTok{"red"}\NormalTok{) }\OperatorTok{+}
\StringTok{  }\KeywordTok{ylim}\NormalTok{(}\DecValTok{0}\NormalTok{,}\DecValTok{20}\NormalTok{) }\OperatorTok{+}
\StringTok{  }\KeywordTok{ggtitle}\NormalTok{(}\StringTok{"GARCH(1,1) Conditional Standard Deviation"}\NormalTok{, }\StringTok{"High Persistence"}\NormalTok{) }\OperatorTok{+}
\StringTok{  }\KeywordTok{xlab}\NormalTok{(}\StringTok{"Index"}\NormalTok{) }\OperatorTok{+}
\StringTok{  }\KeywordTok{ylab}\NormalTok{(}\StringTok{"Value"}\NormalTok{)}
\end{Highlighting}
\end{Shaded}

\includegraphics{homework_5_markdown_files/figure-latex/unnamed-chunk-1-3.pdf}

\begin{Shaded}
\begin{Highlighting}[]
\KeywordTok{ggplot}\NormalTok{(}\KeywordTok{data.frame}\NormalTok{(m1), }\KeywordTok{aes}\NormalTok{(}\DataTypeTok{x=}\KeywordTok{seq}\NormalTok{(}\DataTypeTok{from=}\DecValTok{1}\NormalTok{, }\DataTypeTok{to=}\KeywordTok{length}\NormalTok{(m1)), }\DataTypeTok{y=}\NormalTok{sds2)) }\OperatorTok{+}
\StringTok{  }\KeywordTok{geom_line}\NormalTok{(}\DataTypeTok{color=}\StringTok{"blue"}\NormalTok{) }\OperatorTok{+}
\StringTok{  }\KeywordTok{ylim}\NormalTok{(}\DecValTok{0}\NormalTok{,}\DecValTok{20}\NormalTok{) }\OperatorTok{+}
\StringTok{  }\KeywordTok{ggtitle}\NormalTok{(}\StringTok{"GARCH(1,1) Conditional Standard Deviation"}\NormalTok{, }\StringTok{"Low Persistence"}\NormalTok{) }\OperatorTok{+}
\StringTok{  }\KeywordTok{xlab}\NormalTok{(}\StringTok{"Index"}\NormalTok{) }\OperatorTok{+}
\StringTok{  }\KeywordTok{ylab}\NormalTok{(}\StringTok{"Value"}\NormalTok{)}
\end{Highlighting}
\end{Shaded}

\includegraphics{homework_5_markdown_files/figure-latex/unnamed-chunk-1-4.pdf}

From the simulated time series plots we have a high persistence series
with a persistence parameter of 0.875 and a low persistence series with
a persistence parameter of 0.25. It is clear that the high persistence
series has more volatility clustering and overall volatility, while the
lowe persistence series resembles as MA process.

\begin{Shaded}
\begin{Highlighting}[]
\CommentTok{# Histogram of original time series}
\KeywordTok{ggplot}\NormalTok{(}\KeywordTok{data.frame}\NormalTok{(m1), }\KeywordTok{aes}\NormalTok{(}\DataTypeTok{x=}\NormalTok{m1)) }\OperatorTok{+}
\StringTok{  }\KeywordTok{geom_histogram}\NormalTok{(}\DataTypeTok{fill=}\StringTok{"red"}\NormalTok{, }\DataTypeTok{color=}\StringTok{"black"}\NormalTok{) }\OperatorTok{+}
\StringTok{  }\KeywordTok{xlim}\NormalTok{(}\OperatorTok{-}\DecValTok{21}\NormalTok{, }\DecValTok{21}\NormalTok{) }\OperatorTok{+}
\StringTok{  }\KeywordTok{ggtitle}\NormalTok{(}\StringTok{"Histogram GARCH(1,1)"}\NormalTok{, }\StringTok{"High Persistence"}\NormalTok{) }\OperatorTok{+}
\StringTok{  }\KeywordTok{xlab}\NormalTok{(}\StringTok{"Value"}\NormalTok{) }\OperatorTok{+}
\StringTok{  }\KeywordTok{ylab}\NormalTok{(}\StringTok{"Count"}\NormalTok{) }
\end{Highlighting}
\end{Shaded}

\includegraphics{homework_5_markdown_files/figure-latex/unnamed-chunk-2-1.pdf}

\begin{Shaded}
\begin{Highlighting}[]
\KeywordTok{ggplot}\NormalTok{(}\KeywordTok{data.frame}\NormalTok{(m2), }\KeywordTok{aes}\NormalTok{(}\DataTypeTok{x=}\NormalTok{m2)) }\OperatorTok{+}
\StringTok{  }\KeywordTok{geom_histogram}\NormalTok{(}\DataTypeTok{fill=}\StringTok{"blue"}\NormalTok{, }\DataTypeTok{color=}\StringTok{"black"}\NormalTok{) }\OperatorTok{+}
\StringTok{  }\KeywordTok{xlim}\NormalTok{(}\OperatorTok{-}\DecValTok{21}\NormalTok{, }\DecValTok{21}\NormalTok{) }\OperatorTok{+}
\StringTok{  }\KeywordTok{ggtitle}\NormalTok{(}\StringTok{"Histogram GARCH(1,1)"}\NormalTok{, }\StringTok{"Low Persistence"}\NormalTok{) }\OperatorTok{+}
\StringTok{  }\KeywordTok{xlab}\NormalTok{(}\StringTok{"Value"}\NormalTok{) }\OperatorTok{+}
\StringTok{  }\KeywordTok{ylab}\NormalTok{(}\StringTok{"Count"}\NormalTok{)}
\end{Highlighting}
\end{Shaded}

\includegraphics{homework_5_markdown_files/figure-latex/unnamed-chunk-2-2.pdf}

\begin{Shaded}
\begin{Highlighting}[]
\CommentTok{# Histogram of standardized time series}
\KeywordTok{ggplot}\NormalTok{(}\KeywordTok{data.frame}\NormalTok{(sm1), }\KeywordTok{aes}\NormalTok{(}\DataTypeTok{x=}\NormalTok{sm1)) }\OperatorTok{+}
\StringTok{  }\KeywordTok{geom_histogram}\NormalTok{(}\DataTypeTok{fill=}\StringTok{"red"}\NormalTok{, }\DataTypeTok{color=}\StringTok{"black"}\NormalTok{) }\OperatorTok{+}
\StringTok{  }\KeywordTok{xlim}\NormalTok{(}\OperatorTok{-}\DecValTok{4}\NormalTok{, }\DecValTok{4}\NormalTok{) }\OperatorTok{+}
\StringTok{  }\KeywordTok{ggtitle}\NormalTok{(}\StringTok{"Histogram of Standardized GARCH(1,1)"}\NormalTok{, }\StringTok{"High Persistence"}\NormalTok{) }\OperatorTok{+}
\StringTok{  }\KeywordTok{xlab}\NormalTok{(}\StringTok{"Value"}\NormalTok{) }\OperatorTok{+}
\StringTok{  }\KeywordTok{ylab}\NormalTok{(}\StringTok{"Count"}\NormalTok{) }
\end{Highlighting}
\end{Shaded}

\includegraphics{homework_5_markdown_files/figure-latex/unnamed-chunk-2-3.pdf}

\begin{Shaded}
\begin{Highlighting}[]
\KeywordTok{ggplot}\NormalTok{(}\KeywordTok{data.frame}\NormalTok{(sm2), }\KeywordTok{aes}\NormalTok{(}\DataTypeTok{x=}\NormalTok{sm2)) }\OperatorTok{+}
\StringTok{  }\KeywordTok{geom_histogram}\NormalTok{(}\DataTypeTok{fill=}\StringTok{"blue"}\NormalTok{, }\DataTypeTok{color=}\StringTok{"black"}\NormalTok{) }\OperatorTok{+}
\StringTok{  }\KeywordTok{xlim}\NormalTok{(}\OperatorTok{-}\DecValTok{4}\NormalTok{, }\DecValTok{4}\NormalTok{) }\OperatorTok{+}
\StringTok{  }\KeywordTok{ggtitle}\NormalTok{(}\StringTok{"Histogram of Standardized GARCH(1,1)"}\NormalTok{, }\StringTok{"Low Persistence"}\NormalTok{) }\OperatorTok{+}
\StringTok{  }\KeywordTok{xlab}\NormalTok{(}\StringTok{"Value"}\NormalTok{) }\OperatorTok{+}
\StringTok{  }\KeywordTok{ylab}\NormalTok{(}\StringTok{"Count"}\NormalTok{)}
\end{Highlighting}
\end{Shaded}

\includegraphics{homework_5_markdown_files/figure-latex/unnamed-chunk-2-4.pdf}

The histogram of the high persistence series shows fatter tails than the
low persistence series. This is expected behavior as the persistence
causes high and low extreme values to persist over time and thus have
higher frequencies. The standardized series histograms exhibit the same
patterns but less extremely (because they are standardized).

\begin{Shaded}
\begin{Highlighting}[]
\CommentTok{# ACF/PACF original time series}
\KeywordTok{par}\NormalTok{(}\DataTypeTok{mfrow=}\KeywordTok{c}\NormalTok{(}\DecValTok{1}\NormalTok{,}\DecValTok{2}\NormalTok{))}
\KeywordTok{acf}\NormalTok{(m1, }\DataTypeTok{main=}\StringTok{"ACF - GARCH(1,1) High Persistence"}\NormalTok{)}
\KeywordTok{pacf}\NormalTok{(m1, }\DataTypeTok{main=}\StringTok{"PACF - GARCH(1,1) High Persistence"}\NormalTok{)}
\end{Highlighting}
\end{Shaded}

\includegraphics{homework_5_markdown_files/figure-latex/unnamed-chunk-3-1.pdf}

\begin{Shaded}
\begin{Highlighting}[]
\KeywordTok{par}\NormalTok{(}\DataTypeTok{mfrow=}\KeywordTok{c}\NormalTok{(}\DecValTok{1}\NormalTok{,}\DecValTok{2}\NormalTok{))}
\KeywordTok{acf}\NormalTok{(m2, }\DataTypeTok{main=}\StringTok{"ACF - GARCH(1,1) Low Persistence"}\NormalTok{)}
\KeywordTok{pacf}\NormalTok{(m2, }\DataTypeTok{main=}\StringTok{"PACF - GARCH(1,1) Low Persistence"}\NormalTok{)}
\end{Highlighting}
\end{Shaded}

\includegraphics{homework_5_markdown_files/figure-latex/unnamed-chunk-3-2.pdf}

\begin{Shaded}
\begin{Highlighting}[]
\CommentTok{# ACF/PACF standardized series}
\KeywordTok{par}\NormalTok{(}\DataTypeTok{mfrow=}\KeywordTok{c}\NormalTok{(}\DecValTok{1}\NormalTok{,}\DecValTok{2}\NormalTok{))}
\KeywordTok{acf}\NormalTok{(sm1, }\DataTypeTok{main=}\StringTok{"ACF - GARCH(1,1) Standardized, High Persistence"}\NormalTok{)}
\KeywordTok{pacf}\NormalTok{(sm1, }\DataTypeTok{main=}\StringTok{"PACF - GARCH(1,1) Standardized, High Persistence"}\NormalTok{)}
\end{Highlighting}
\end{Shaded}

\includegraphics{homework_5_markdown_files/figure-latex/unnamed-chunk-3-3.pdf}

\begin{Shaded}
\begin{Highlighting}[]
\KeywordTok{par}\NormalTok{(}\DataTypeTok{mfrow=}\KeywordTok{c}\NormalTok{(}\DecValTok{1}\NormalTok{,}\DecValTok{2}\NormalTok{))}
\KeywordTok{acf}\NormalTok{(sm2, }\DataTypeTok{main=}\StringTok{"ACF - GARCH(1,1) Standardized, Low Persistence"}\NormalTok{)}
\KeywordTok{pacf}\NormalTok{(sm2, }\DataTypeTok{main=}\StringTok{"PACF - GARCH(1,1) Standardized, Low Persistence"}\NormalTok{)}
\end{Highlighting}
\end{Shaded}

\includegraphics{homework_5_markdown_files/figure-latex/unnamed-chunk-3-4.pdf}

The ACF and PACF's of the original series and standardized original
series dont really show any time dependence besides up to a lag of 1 in
the high persistence series.

\begin{Shaded}
\begin{Highlighting}[]
\CommentTok{# ACF/PACF squared time series}
\KeywordTok{par}\NormalTok{(}\DataTypeTok{mfrow=}\KeywordTok{c}\NormalTok{(}\DecValTok{1}\NormalTok{,}\DecValTok{2}\NormalTok{))}
\KeywordTok{acf}\NormalTok{(m1}\OperatorTok{^}\DecValTok{2}\NormalTok{, }\DataTypeTok{main=}\StringTok{"ACF - GARCH(1,1) Squared, High Persistence"}\NormalTok{)}
\KeywordTok{pacf}\NormalTok{(m1}\OperatorTok{^}\DecValTok{2}\NormalTok{, }\DataTypeTok{main=}\StringTok{"PACF - GARCH(1,1) Squared, High Persistence"}\NormalTok{)}
\end{Highlighting}
\end{Shaded}

\includegraphics{homework_5_markdown_files/figure-latex/unnamed-chunk-4-1.pdf}

\begin{Shaded}
\begin{Highlighting}[]
\KeywordTok{par}\NormalTok{(}\DataTypeTok{mfrow=}\KeywordTok{c}\NormalTok{(}\DecValTok{1}\NormalTok{,}\DecValTok{2}\NormalTok{))}
\KeywordTok{acf}\NormalTok{(m2}\OperatorTok{^}\DecValTok{2}\NormalTok{, }\DataTypeTok{main=}\StringTok{"ACF - GARCH(1,1) Squared, Low Persistence"}\NormalTok{)}
\KeywordTok{pacf}\NormalTok{(m2}\OperatorTok{^}\DecValTok{2}\NormalTok{, }\DataTypeTok{main=}\StringTok{"PACF - GARCH(1,1) Squared, Low Persistence"}\NormalTok{)}
\end{Highlighting}
\end{Shaded}

\includegraphics{homework_5_markdown_files/figure-latex/unnamed-chunk-4-2.pdf}

\begin{Shaded}
\begin{Highlighting}[]
\CommentTok{# ACF/PACF squared standardized series}
\KeywordTok{par}\NormalTok{(}\DataTypeTok{mfrow=}\KeywordTok{c}\NormalTok{(}\DecValTok{1}\NormalTok{,}\DecValTok{2}\NormalTok{))}
\KeywordTok{acf}\NormalTok{(sm1}\OperatorTok{^}\DecValTok{2}\NormalTok{, }\DataTypeTok{main=}\StringTok{"ACF - GARCH(1,1) Standardized Squared, High Persistence"}\NormalTok{)}
\KeywordTok{pacf}\NormalTok{(sm1}\OperatorTok{^}\DecValTok{2}\NormalTok{, }\DataTypeTok{main=}\StringTok{"PACF - GARCH(1,1) Standardized Squared, High Persistence"}\NormalTok{)}
\end{Highlighting}
\end{Shaded}

\includegraphics{homework_5_markdown_files/figure-latex/unnamed-chunk-4-3.pdf}

\begin{Shaded}
\begin{Highlighting}[]
\KeywordTok{par}\NormalTok{(}\DataTypeTok{mfrow=}\KeywordTok{c}\NormalTok{(}\DecValTok{1}\NormalTok{,}\DecValTok{2}\NormalTok{))}
\KeywordTok{acf}\NormalTok{(sm2}\OperatorTok{^}\DecValTok{2}\NormalTok{, }\DataTypeTok{main=}\StringTok{"ACF - GARCH(1,1) Standardized Squared, Low Persistence"}\NormalTok{)}
\KeywordTok{pacf}\NormalTok{(sm2}\OperatorTok{^}\DecValTok{2}\NormalTok{, }\DataTypeTok{main=}\StringTok{"PACF - GARCH(1,1) Standardized Squared, Low Persistence"}\NormalTok{)}
\end{Highlighting}
\end{Shaded}

\includegraphics{homework_5_markdown_files/figure-latex/unnamed-chunk-4-4.pdf}

Now, the squared series shows much more time dependence. This is because
squaring the series essentially shows us the magnitude of variation from
the expected value. Now we see the high persistence series showing some
strong time dependence with a slow decay to zero. The low persistence
series show only dependence up to one lag.

\subsection{Problem 14.3}\label{problem-14.3}

\begin{Shaded}
\begin{Highlighting}[]
\NormalTok{sp <-}\StringTok{ }\KeywordTok{read_xls}\NormalTok{(}\StringTok{"/Users/noahkawasaki/Desktop/ECON 144/Textbook_data/Chapter14_exercises_data.xls"}\NormalTok{)}
\NormalTok{df <-}\StringTok{ }\KeywordTok{data.frame}\NormalTok{(sp}\OperatorTok{$}\NormalTok{Date, sp}\OperatorTok{$}\StringTok{`}\DataTypeTok{Adj Close}\StringTok{`}\NormalTok{) }\OperatorTok
\StringTok{  }\KeywordTok{set_names}\NormalTok{(}\StringTok{"date"}\NormalTok{, }\StringTok{"adjusted"}\NormalTok{) }\OperatorTok
\StringTok{  }\KeywordTok{mutate}\NormalTok{(}
    \DataTypeTok{return =} \KeywordTok{log}\NormalTok{(adjusted) }\OperatorTok{-}\StringTok{ }\KeywordTok{log}\NormalTok{(}\KeywordTok{lag}\NormalTok{(adjusted))}
\NormalTok{  )}
\NormalTok{df <-}\StringTok{ }\KeywordTok{na.omit}\NormalTok{(df)}


\CommentTok{# Plot time series and return series}
\KeywordTok{ggplot}\NormalTok{(df, }\KeywordTok{aes}\NormalTok{(}\DataTypeTok{x=}\NormalTok{date, }\DataTypeTok{y=}\NormalTok{adjusted)) }\OperatorTok{+}
\StringTok{  }\KeywordTok{geom_line}\NormalTok{(}\DataTypeTok{color=}\StringTok{"red"}\NormalTok{) }\OperatorTok{+}
\StringTok{  }\KeywordTok{ggtitle}\NormalTok{(}\StringTok{"S&P500 Price"}\NormalTok{, }\StringTok{"2000-2013"}\NormalTok{) }\OperatorTok{+}
\StringTok{  }\KeywordTok{xlab}\NormalTok{(}\StringTok{"Date"}\NormalTok{) }\OperatorTok{+}
\StringTok{  }\KeywordTok{ylab}\NormalTok{(}\StringTok{"Price"}\NormalTok{)}
\end{Highlighting}
\end{Shaded}

\includegraphics{homework_5_markdown_files/figure-latex/unnamed-chunk-5-1.pdf}

\begin{Shaded}
\begin{Highlighting}[]
\KeywordTok{ggplot}\NormalTok{(df, }\KeywordTok{aes}\NormalTok{(}\DataTypeTok{x=}\NormalTok{date, }\DataTypeTok{y=}\NormalTok{return)) }\OperatorTok{+}
\StringTok{  }\KeywordTok{geom_line}\NormalTok{(}\DataTypeTok{color=}\StringTok{"red"}\NormalTok{, }\DataTypeTok{na.rm=}\OtherTok{TRUE}\NormalTok{) }\OperatorTok{+}
\StringTok{  }\KeywordTok{ggtitle}\NormalTok{(}\StringTok{"S&P500 Returns"}\NormalTok{, }\StringTok{"2000-2013"}\NormalTok{) }\OperatorTok{+}
\StringTok{  }\KeywordTok{xlab}\NormalTok{(}\StringTok{"Date"}\NormalTok{) }\OperatorTok{+}
\StringTok{  }\KeywordTok{ylab}\NormalTok{(}\StringTok{"Rate"}\NormalTok{)}
\end{Highlighting}
\end{Shaded}

\includegraphics{homework_5_markdown_files/figure-latex/unnamed-chunk-5-2.pdf}

From the time series and returns plot of the S\&P 500 it is obvious that
this data has periods of high volatility and low volatility. Notably,
from 2004 to 2007 is a period of low variation, but from 2007 to 2010
there is some extreme deviation behavior.

\begin{Shaded}
\begin{Highlighting}[]
\CommentTok{# ACF and PACFs}
\KeywordTok{par}\NormalTok{(}\DataTypeTok{mfrow=}\KeywordTok{c}\NormalTok{(}\DecValTok{1}\NormalTok{,}\DecValTok{2}\NormalTok{))}
\KeywordTok{acf}\NormalTok{(df}\OperatorTok{$}\NormalTok{return, }\DataTypeTok{main=}\StringTok{"ACF - S&P500 Returns"}\NormalTok{)}
\KeywordTok{pacf}\NormalTok{(df}\OperatorTok{$}\NormalTok{return, }\DataTypeTok{main=}\StringTok{"PACF - S&P500 Returns"}\NormalTok{)}
\end{Highlighting}
\end{Shaded}

\includegraphics{homework_5_markdown_files/figure-latex/unnamed-chunk-6-1.pdf}

\begin{Shaded}
\begin{Highlighting}[]
\KeywordTok{par}\NormalTok{(}\DataTypeTok{mfrow=}\KeywordTok{c}\NormalTok{(}\DecValTok{1}\NormalTok{,}\DecValTok{2}\NormalTok{))}
\KeywordTok{acf}\NormalTok{((df}\OperatorTok{$}\NormalTok{return)}\OperatorTok{^}\DecValTok{2}\NormalTok{, }\DataTypeTok{main=}\StringTok{"ACF - S&P500 Squared Returns"}\NormalTok{)}
\KeywordTok{pacf}\NormalTok{((df}\OperatorTok{$}\NormalTok{return)}\OperatorTok{^}\DecValTok{2}\NormalTok{, }\DataTypeTok{main=}\StringTok{"PACF - S&P500 Squared Returns"}\NormalTok{)}
\end{Highlighting}
\end{Shaded}

\includegraphics{homework_5_markdown_files/figure-latex/unnamed-chunk-6-2.pdf}

If we look at the ACF and PACF's of the original returns and the squared
returns we'll see that there is some variance time dependence, and that
ARCH and GARCH models should be applied to tease out these signals.

\begin{Shaded}
\begin{Highlighting}[]
\CommentTok{# Find best ARCH Model}
\NormalTok{arch =}\StringTok{ }\KeywordTok{garch}\NormalTok{(df}\OperatorTok{$}\NormalTok{return, }\DataTypeTok{order=}\KeywordTok{c}\NormalTok{(}\DecValTok{0}\NormalTok{,}\DecValTok{11}\NormalTok{), }\DataTypeTok{trace=}\NormalTok{F)}
\KeywordTok{summary}\NormalTok{(arch)}
\end{Highlighting}
\end{Shaded}

\begin{verbatim}
## 
## Call:
## garch(x = df$return, order = c(0, 11), trace = F)
## 
## Model:
## GARCH(0,11)
## 
## Residuals:
##      Min       1Q   Median       3Q      Max 
## -5.66957 -0.56424  0.05792  0.61311  3.23808 
## 
## Coefficient(s):
##      Estimate  Std. Error  t value Pr(>|t|)    
## a0  2.089e-05   1.900e-06   10.994  < 2e-16 ***
## a1  1.117e-02   1.022e-02    1.093 0.274326    
## a2  1.369e-01   1.545e-02    8.862  < 2e-16 ***
## a3  8.298e-02   1.836e-02    4.519 6.22e-06 ***
## a4  9.378e-02   1.858e-02    5.046 4.51e-07 ***
## a5  9.358e-02   1.924e-02    4.865 1.15e-06 ***
## a6  5.279e-02   1.877e-02    2.812 0.004929 ** 
## a7  7.800e-02   1.595e-02    4.889 1.01e-06 ***
## a8  1.172e-01   2.033e-02    5.762 8.29e-09 ***
## a9  6.362e-02   1.731e-02    3.674 0.000239 ***
## a10 8.660e-02   1.889e-02    4.584 4.56e-06 ***
## a11 7.378e-02   1.874e-02    3.936 8.28e-05 ***
## ---
## Signif. codes:  0 '***' 0.001 '**' 0.01 '*' 0.05 '.' 0.1 ' ' 1
## 
## Diagnostic Tests:
##  Jarque Bera Test
## 
## data:  Residuals
## X-squared = 203.59, df = 2, p-value < 2.2e-16
## 
## 
##  Box-Ljung test
## 
## data:  Squared.Residuals
## X-squared = 0.03471, df = 1, p-value = 0.8522
\end{verbatim}

\begin{Shaded}
\begin{Highlighting}[]
\KeywordTok{logLik}\NormalTok{(arch)}
\end{Highlighting}
\end{Shaded}

\begin{verbatim}
## 'log Lik.' 10453.14 (df=12)
\end{verbatim}

The best ARCH model for this data series is a high order one at
ARCH(11). All coefficients but the a1 are statistically significant, and
the model achieves a log likelihood of 10453.

\begin{Shaded}
\begin{Highlighting}[]
\KeywordTok{par}\NormalTok{(}\DataTypeTok{mfrow=}\KeywordTok{c}\NormalTok{(}\DecValTok{1}\NormalTok{,}\DecValTok{2}\NormalTok{))}
\KeywordTok{acf}\NormalTok{(arch}\OperatorTok{$}\NormalTok{residuals[}\DecValTok{12}\OperatorTok{:}\KeywordTok{length}\NormalTok{(arch}\OperatorTok{$}\NormalTok{residuals)], }\DataTypeTok{main=}\StringTok{"ACF - ARCH(11) Residuals"}\NormalTok{)}
\KeywordTok{pacf}\NormalTok{(arch}\OperatorTok{$}\NormalTok{residuals[}\DecValTok{12}\OperatorTok{:}\KeywordTok{length}\NormalTok{(arch}\OperatorTok{$}\NormalTok{residuals)], }\DataTypeTok{main=}\StringTok{"PACF - ARCH(11) Residuals"}\NormalTok{)}
\end{Highlighting}
\end{Shaded}

\includegraphics{homework_5_markdown_files/figure-latex/unnamed-chunk-8-1.pdf}

Looking at the ACF and PACF of the residuals, it appears that for the
most part these signals have been accounted for.

\begin{Shaded}
\begin{Highlighting}[]
\CommentTok{# GARCH Model}
\NormalTok{garch <-}\StringTok{ }\KeywordTok{garchFit}\NormalTok{(}\OperatorTok{~}\KeywordTok{garch}\NormalTok{(}\DecValTok{2}\NormalTok{,}\DecValTok{2}\NormalTok{), }\DataTypeTok{data=}\NormalTok{df}\OperatorTok{$}\NormalTok{return, }\DataTypeTok{trace=}\OtherTok{FALSE}\NormalTok{)}
\KeywordTok{summary}\NormalTok{(garch)}
\end{Highlighting}
\end{Shaded}

\begin{verbatim}
## 
## Title:
##  GARCH Modelling 
## 
## Call:
##  garchFit(formula = ~garch(2, 2), data = df$return, trace = FALSE) 
## 
## Mean and Variance Equation:
##  data ~ garch(2, 2)
## <environment: 0x7f81293841f0>
##  [data = df$return]
## 
## Conditional Distribution:
##  norm 
## 
## Coefficient(s):
##         mu       omega      alpha1      alpha2       beta1       beta2  
## 3.1522e-04  2.6470e-06  1.2701e-02  1.2445e-01  5.5700e-01  2.8875e-01  
## 
## Std. Errors:
##  based on Hessian 
## 
## Error Analysis:
##         Estimate  Std. Error  t value Pr(>|t|)    
## mu     3.152e-04   1.530e-04    2.060  0.03943 *  
## omega  2.647e-06   5.786e-07    4.575 4.76e-06 ***
## alpha1 1.270e-02   1.298e-02    0.978  0.32793    
## alpha2 1.245e-01   2.060e-02    6.040 1.54e-09 ***
## beta1  5.570e-01   1.718e-01    3.241  0.00119 ** 
## beta2  2.887e-01   1.575e-01    1.834  0.06669 .  
## ---
## Signif. codes:  0 '***' 0.001 '**' 0.01 '*' 0.05 '.' 0.1 ' ' 1
## 
## Log Likelihood:
##  10489.13    normalized:  3.127351 
## 
## Description:
##  Thu Jun  7 14:40:31 2018 by user:  
## 
## 
## Standardised Residuals Tests:
##                                 Statistic p-Value     
##  Jarque-Bera Test   R    Chi^2  238.1684  0           
##  Shapiro-Wilk Test  R    W      0.9892104 2.722248e-15
##  Ljung-Box Test     R    Q(10)  22.1942   0.01414527  
##  Ljung-Box Test     R    Q(15)  32.07282  0.006292776 
##  Ljung-Box Test     R    Q(20)  38.11685  0.00856893  
##  Ljung-Box Test     R^2  Q(10)  6.155105  0.8020679   
##  Ljung-Box Test     R^2  Q(15)  9.255884  0.8637624   
##  Ljung-Box Test     R^2  Q(20)  12.06721  0.9137434   
##  LM Arch Test       R    TR^2   6.430372  0.8928536   
## 
## Information Criterion Statistics:
##       AIC       BIC       SIC      HQIC 
## -6.251124 -6.240180 -6.251130 -6.247210
\end{verbatim}

While the ARCH(11) did a good job at modeling the S\&P returns, it
contains 12 parameters to estimate. So we can also fit a GARCH model to
achieve similar (better?) results with less computing. The chosen model
is GARCH(2,2). Its coefficients are statistically significant besides
the alpha1, and gets a higher log likelihood of 10487.

\begin{Shaded}
\begin{Highlighting}[]
\KeywordTok{par}\NormalTok{(}\DataTypeTok{mfrow=}\KeywordTok{c}\NormalTok{(}\DecValTok{1}\NormalTok{,}\DecValTok{2}\NormalTok{))}
\KeywordTok{acf}\NormalTok{(garch}\OperatorTok{@}\NormalTok{residuals, }\DataTypeTok{main=}\StringTok{"ACF - GARCH(2,2) Residuals"}\NormalTok{)}
\KeywordTok{pacf}\NormalTok{(garch}\OperatorTok{@}\NormalTok{residuals, }\DataTypeTok{main=}\StringTok{"ACF - GARCH(2,2) Residuals"}\NormalTok{)}
\end{Highlighting}
\end{Shaded}

\includegraphics{homework_5_markdown_files/figure-latex/unnamed-chunk-10-1.pdf}

The ACF and PACF of the GARCH(2,2) residuals also suggest that the model
effectively captured the volatile behavior.

\subsection{Problem 14.4}\label{problem-14.4}

\begin{Shaded}
\begin{Highlighting}[]
\CommentTok{# One and two step ahead forecasts}
\NormalTok{predicts <-}\StringTok{ }\KeywordTok{predict}\NormalTok{(garch, }\DataTypeTok{n.ahead=}\DecValTok{2}\NormalTok{, }\DataTypeTok{plot=}\OtherTok{TRUE}\NormalTok{)}
\end{Highlighting}
\end{Shaded}

\includegraphics{homework_5_markdown_files/figure-latex/unnamed-chunk-11-1.pdf}

\begin{Shaded}
\begin{Highlighting}[]
\NormalTok{predicts}
\end{Highlighting}
\end{Shaded}

\begin{verbatim}
##   meanForecast   meanError standardDeviation lowerInterval upperInterval
## 1 0.0003152196 0.008928532       0.008928532   -0.01718438    0.01781482
## 2 0.0003152196 0.008387256       0.008387256   -0.01612350    0.01675394
\end{verbatim}

\subsection{Textbook B}\label{textbook-b}

\subsection{Problem 14.1}\label{problem-14.1}

\begin{Shaded}
\begin{Highlighting}[]
\NormalTok{## a}
\KeywordTok{getSymbols}\NormalTok{(}\StringTok{"^NYA"}\NormalTok{, }\DataTypeTok{from=}\StringTok{"1988-01-01"}\NormalTok{, }\DataTypeTok{to=}\StringTok{"2001-12-31"}\NormalTok{)}
\end{Highlighting}
\end{Shaded}

\begin{verbatim}
## [1] "NYA"
\end{verbatim}

\begin{Shaded}
\begin{Highlighting}[]
\NormalTok{nyse <-}\StringTok{ }\KeywordTok{data.frame}\NormalTok{(}\KeywordTok{log}\NormalTok{(NYA}\OperatorTok{$}\NormalTok{NYA.Adjusted) }\OperatorTok{-}\StringTok{ }\KeywordTok{log}\NormalTok{(}\KeywordTok{lag}\NormalTok{(NYA}\OperatorTok{$}\NormalTok{NYA.Adjusted))[}\OperatorTok{-}\DecValTok{1}\NormalTok{])}
\NormalTok{nyse <-}\StringTok{ }\NormalTok{tibble}\OperatorTok{::}\KeywordTok{rownames_to_column}\NormalTok{(nyse) }\OperatorTok
\StringTok{  }\KeywordTok{set_names}\NormalTok{(}\StringTok{"date"}\NormalTok{, }\StringTok{"return"}\NormalTok{)}

\NormalTok{train <-}\StringTok{ }\NormalTok{nyse[}\DecValTok{1}\OperatorTok{:}\DecValTok{3461}\NormalTok{, ]}
\NormalTok{test <-}\StringTok{ }\NormalTok{nyse[}\DecValTok{3462}\OperatorTok{:}\KeywordTok{nrow}\NormalTok{(nyse), ]}

\NormalTok{ts <-}\StringTok{ }\KeywordTok{ts}\NormalTok{(train}\OperatorTok{$}\NormalTok{return)}
\NormalTok{ts_test <-}\StringTok{ }\KeywordTok{ts}\NormalTok{(test}\OperatorTok{$}\NormalTok{return)}

\CommentTok{# Plot}
\KeywordTok{ggplot}\NormalTok{(train, }\KeywordTok{aes}\NormalTok{(}\DataTypeTok{x=}\KeywordTok{seq}\NormalTok{(}\DataTypeTok{from=}\DecValTok{1}\NormalTok{, }\DataTypeTok{to=}\KeywordTok{length}\NormalTok{(return)), }\DataTypeTok{y=}\NormalTok{return, }\DataTypeTok{group=}\DecValTok{1}\NormalTok{)) }\OperatorTok{+}
\StringTok{  }\KeywordTok{geom_line}\NormalTok{(}\DataTypeTok{color=}\StringTok{"green"}\NormalTok{) }\OperatorTok{+}
\StringTok{  }\KeywordTok{ggtitle}\NormalTok{(}\StringTok{"NYSE Daily Returns"}\NormalTok{, }\StringTok{"1988-2001"}\NormalTok{) }\OperatorTok{+}
\StringTok{  }\KeywordTok{xlab}\NormalTok{(}\StringTok{"Index"}\NormalTok{) }\OperatorTok{+}
\StringTok{  }\KeywordTok{ylab}\NormalTok{(}\StringTok{"Rate"}\NormalTok{)}
\end{Highlighting}
\end{Shaded}

\includegraphics{homework_5_markdown_files/figure-latex/unnamed-chunk-12-1.pdf}

\begin{Shaded}
\begin{Highlighting}[]
\CommentTok{# AR}
\NormalTok{m =}\StringTok{ }\KeywordTok{auto.arima}\NormalTok{(ts)}
\KeywordTok{summary}\NormalTok{(m)}
\end{Highlighting}
\end{Shaded}

\begin{verbatim}
## Series: ts 
## ARIMA(5,0,3) with non-zero mean 
## 
## Coefficients:
##          ar1     ar2      ar3     ar4      ar5      ma1      ma2     ma3
##       0.8503  0.3857  -0.6647  0.0736  -0.0526  -0.8023  -0.4509  0.6292
## s.e.  0.1601  0.2409   0.1552  0.0244   0.0232   0.1599   0.2360  0.1454
##        mean
##       4e-04
## s.e.  1e-04
## 
## sigma^2 estimated as 7.241e-05:  log likelihood=11590.62
## AIC=-23161.25   AICc=-23161.18   BIC=-23099.75
## 
## Training set error measures:
##                        ME        RMSE        MAE MPE MAPE      MASE
## Training set -1.62577e-06 0.008498468 0.00598073 NaN  Inf 0.7023232
##                       ACF1
## Training set -0.0001363117
\end{verbatim}

\begin{Shaded}
\begin{Highlighting}[]
\NormalTok{resids <-}\StringTok{ }\NormalTok{m}\OperatorTok{$}\NormalTok{residuals[}\DecValTok{13}\OperatorTok{:}\KeywordTok{length}\NormalTok{(m}\OperatorTok{$}\NormalTok{residuals)]}

\KeywordTok{par}\NormalTok{(}\DataTypeTok{mfrow=}\KeywordTok{c}\NormalTok{(}\DecValTok{1}\NormalTok{,}\DecValTok{2}\NormalTok{))}
\KeywordTok{acf}\NormalTok{(resids)}
\KeywordTok{pacf}\NormalTok{(resids)}
\end{Highlighting}
\end{Shaded}

\includegraphics{homework_5_markdown_files/figure-latex/unnamed-chunk-12-2.pdf}

\begin{Shaded}
\begin{Highlighting}[]
\KeywordTok{par}\NormalTok{(}\DataTypeTok{mfrow=}\KeywordTok{c}\NormalTok{(}\DecValTok{1}\NormalTok{,}\DecValTok{2}\NormalTok{))}
\KeywordTok{acf}\NormalTok{((resids)}\OperatorTok{^}\DecValTok{2}\NormalTok{)}
\KeywordTok{pacf}\NormalTok{((resids)}\OperatorTok{^}\DecValTok{2}\NormalTok{)}
\end{Highlighting}
\end{Shaded}

\includegraphics{homework_5_markdown_files/figure-latex/unnamed-chunk-12-3.pdf}

\begin{Shaded}
\begin{Highlighting}[]
\CommentTok{# GARCH}
\NormalTok{garch <-}\StringTok{ }\KeywordTok{garchFit}\NormalTok{(}\OperatorTok{~}\KeywordTok{garch}\NormalTok{(}\DecValTok{1}\NormalTok{,}\DecValTok{2}\NormalTok{), }\DataTypeTok{data=}\NormalTok{resids, }\DataTypeTok{trace=}\OtherTok{FALSE}\NormalTok{)}
\KeywordTok{summary}\NormalTok{(garch)}
\end{Highlighting}
\end{Shaded}

\begin{verbatim}
## 
## Title:
##  GARCH Modelling 
## 
## Call:
##  garchFit(formula = ~garch(1, 2), data = resids, trace = FALSE) 
## 
## Mean and Variance Equation:
##  data ~ garch(1, 2)
## <environment: 0x7f812d139d80>
##  [data = resids]
## 
## Conditional Distribution:
##  norm 
## 
## Coefficient(s):
##         mu       omega      alpha1       beta1       beta2  
## 1.0776e-04  6.8337e-07  5.1809e-02  9.1626e-01  2.4135e-02  
## 
## Std. Errors:
##  based on Hessian 
## 
## Error Analysis:
##         Estimate  Std. Error  t value Pr(>|t|)    
## mu     1.078e-04   1.213e-04    0.889  0.37420    
## omega  6.834e-07   2.300e-07    2.971  0.00297 ** 
## alpha1 5.181e-02   1.020e-02    5.082 3.74e-07 ***
## beta1  9.163e-01   1.545e-01    5.930 3.03e-09 ***
## beta2  2.414e-02   1.489e-01    0.162  0.87123    
## ---
## Signif. codes:  0 '***' 0.001 '**' 0.01 '*' 0.05 '.' 0.1 ' ' 1
## 
## Log Likelihood:
##  11860.31    normalized:  3.438768 
## 
## Description:
##  Thu Jun  7 14:40:37 2018 by user:  
## 
## 
## Standardised Residuals Tests:
##                                 Statistic p-Value  
##  Jarque-Bera Test   R    Chi^2  5229.4    0        
##  Shapiro-Wilk Test  R    W      0.9595893 0        
##  Ljung-Box Test     R    Q(10)  10.20268  0.4228949
##  Ljung-Box Test     R    Q(15)  15.18456  0.4382065
##  Ljung-Box Test     R    Q(20)  18.0473   0.584292 
##  Ljung-Box Test     R^2  Q(10)  3.603135  0.9634799
##  Ljung-Box Test     R^2  Q(15)  5.478519  0.9872411
##  Ljung-Box Test     R^2  Q(20)  6.373636  0.9982896
##  LM Arch Test       R    TR^2   4.75903   0.9655469
## 
## Information Criterion Statistics:
##       AIC       BIC       SIC      HQIC 
## -6.874637 -6.865728 -6.874642 -6.871455
\end{verbatim}

\begin{Shaded}
\begin{Highlighting}[]
\KeywordTok{par}\NormalTok{(}\DataTypeTok{mfrow=}\KeywordTok{c}\NormalTok{(}\DecValTok{1}\NormalTok{,}\DecValTok{2}\NormalTok{))}
\KeywordTok{acf}\NormalTok{((garch}\OperatorTok{@}\NormalTok{residuals)}\OperatorTok{^}\DecValTok{2}\NormalTok{)}
\KeywordTok{pacf}\NormalTok{((garch}\OperatorTok{@}\NormalTok{residuals)}\OperatorTok{^}\DecValTok{2}\NormalTok{)}
\end{Highlighting}
\end{Shaded}

\includegraphics{homework_5_markdown_files/figure-latex/unnamed-chunk-12-4.pdf}

\subsection{Problem 14.4}\label{problem-14.4-1}

\subsubsection{a)}\label{a}

\begin{Shaded}
\begin{Highlighting}[]
\NormalTok{ar5 <-}\StringTok{ }\KeywordTok{arma}\NormalTok{(ts, }\DataTypeTok{order=}\KeywordTok{c}\NormalTok{(}\DecValTok{5}\NormalTok{,}\DecValTok{0}\NormalTok{))}
\KeywordTok{plot}\NormalTok{(ar5}\OperatorTok{$}\NormalTok{fitted.values)}
\end{Highlighting}
\end{Shaded}

\includegraphics{homework_5_markdown_files/figure-latex/unnamed-chunk-13-1.pdf}

\subsubsection{b)}\label{b}

\begin{Shaded}
\begin{Highlighting}[]
\NormalTok{smooth <-}\StringTok{ }\KeywordTok{ets}\NormalTok{(ts, }\DataTypeTok{model=}\StringTok{"ANN"}\NormalTok{, }\DataTypeTok{alpha=}\FloatTok{0.10}\NormalTok{)}
\KeywordTok{plot}\NormalTok{(smooth)}
\end{Highlighting}
\end{Shaded}

\includegraphics{homework_5_markdown_files/figure-latex/unnamed-chunk-14-1.pdf}

\subsubsection{c)}\label{c}

\begin{Shaded}
\begin{Highlighting}[]
\NormalTok{garch11 <-}\StringTok{ }\KeywordTok{garchFit}\NormalTok{(}\OperatorTok{~}\KeywordTok{garch}\NormalTok{(}\DecValTok{1}\NormalTok{,}\DecValTok{2}\NormalTok{), }\DataTypeTok{data=}\NormalTok{ts, }\DataTypeTok{trace=}\OtherTok{FALSE}\NormalTok{)}
\KeywordTok{plot}\NormalTok{(garch11, }\DataTypeTok{which=}\DecValTok{1}\NormalTok{)}
\end{Highlighting}
\end{Shaded}

\includegraphics{homework_5_markdown_files/figure-latex/unnamed-chunk-15-1.pdf}

\subsubsection{d)}\label{d}

For the most part, visually these models all look very similar. However
the ARCH and GARCH model's are based on actual time dependence as
opposed to the mathematical strategies used in exponential smoothing.
This means that the GARCH's smoothing parameter is chosen through
likelihood optimization, whereas we just chose an arbitrary value for
the ets model.

\subsection{Problem 14.5}\label{problem-14.5}

\subsubsection{a)}\label{a-1}

\begin{Shaded}
\begin{Highlighting}[]
\NormalTok{spec <-}\StringTok{ }\KeywordTok{garchSpec}\NormalTok{(}\DataTypeTok{model=}\KeywordTok{list}\NormalTok{(}\DataTypeTok{mu=}\DecValTok{76}\NormalTok{, }\DataTypeTok{alpha=}\FloatTok{0.6}\NormalTok{, }\DataTypeTok{beta=}\DecValTok{0}\NormalTok{, }\DataTypeTok{omega=}\DecValTok{3}\NormalTok{))}
\NormalTok{sim <-}\StringTok{ }\KeywordTok{garchSim}\NormalTok{(spec, }\DataTypeTok{n=}\DecValTok{100}\NormalTok{)}

\NormalTok{fit <-}\StringTok{ }\KeywordTok{garchFit}\NormalTok{(}\OperatorTok{~}\KeywordTok{garch}\NormalTok{(}\DecValTok{1}\NormalTok{, }\DecValTok{1}\NormalTok{), sim, }\DataTypeTok{trace=}\OtherTok{FALSE}\NormalTok{)}
\KeywordTok{predict}\NormalTok{(fit, }\DataTypeTok{n.ahead=}\DecValTok{10}\NormalTok{)}
\end{Highlighting}
\end{Shaded}

\begin{verbatim}
##    meanForecast meanError standardDeviation
## 1      76.19167  1.854680          1.854680
## 2      76.19167  2.051320          2.051320
## 3      76.19167  2.163883          2.163883
## 4      76.19167  2.230583          2.230583
## 5      76.19167  2.270807          2.270807
## 6      76.19167  2.295304          2.295304
## 7      76.19167  2.310307          2.310307
## 8      76.19167  2.319526          2.319526
## 9      76.19167  2.325203          2.325203
## 10     76.19167  2.328703          2.328703
\end{verbatim}

\subsubsection{b)}\label{b-1}

\subsubsection{c)}\label{c-1}

Answer is not necessarily. Because a GARCH model is based on the time
dependence of volatility and is short term, in order to predict a next
spell of bad weather that bad weather would have already needed to
start.


\end{document}
